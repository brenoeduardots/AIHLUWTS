\chapter{Sparse Matrices Representation with the CSC Method}
\label{appendices_smr}

Appendix \ref{appendices_vis} shows how the mathematical and numerical formulations of this project spawn a sparse system of equations. This appendix shows how the coefficient matrix of this system of equations has been compressed for solving it with linear algebra tools. The first section describes and exemplifies the compression method utilized, the CSC (Compressed Sparse Column), from a general point of view. The last section shows how this method has been utilized in this specific project.

\section{Representation of a Generic Matrix with the CSC Method}

According to \cite{Golub1996}, a matrix is sparse if most of its elements are zero. Figure \ref{fig:27} is an example of one of them. By contrast, if most of the elements in a matrix are non-zero, it is a dense one. Another form of visualizing this is by the concept of sparsity. The sparsity is defined as the number of non-zero entries in the matrix divided by the total quantity of elements in it. If the sparsity of a matrix is higher than 50\%, it is sparse, otherwise, the matrix is dense.

Sparse matrices are related to systems with few pairwise interactions. Since each grid block of the model only has pairwise interactions with their neighbors, the resulting linear system of equations has a sparse coefficient matrix. If all the cells in the grid had pairwise interactions with all the other grid blocks, the resulting matrix would be dense, which is not the case of this problem.

A simple way to computationally store a $m \times n$ matrix $M$ is by utilizing a two-dimensional array, with dimensions $m \times n$. If this matrix is composed of real numbers, those two-dimensional arrays would store $m \times n$ floating-point variables. This is a common way to store data for simple, dense matrices. However, if $M$ is a sparse matrix there are more efficient ways for doing that, avoiding unnecessary storage of the zeroes. For this project, the sparse matrix has been stored with the CSC (Compressed Sparse Column) method. This method is compatible with most of the tools available for solving sparse linear systems, such as UMFPACK and Eigen. For understanding the CSC method, consider below the example shown in the \cite{Davis1995}. A matrix $A$ is defined as:
\begin{equation}
	A = 
	\begin{bmatrix}
	2	&3	&0	&0	&0\\
	3	&0	&4	&0	&6\\
	0	&-1	&-3	&2	&0\\
	0	&0	&1	&0	&0\\
	0	&4	&2	&0	&1
	\end{bmatrix}
	.
\end{equation}
The CSC method consists of storing this matrix by utilizing three arrays: $A_p$, $A_i$, and $A_x$. 

$A_p$ stores the number of non-zero elements in each column, in a specific form. By definition:
\begin{equation}
	A_p[0] = 0,
\end{equation}
and
\nomenclature{$NZCount$}{Count of non-zero elements for a given column of a matrix $M$}
\begin{equation}
A_p[ind] = A_p[ind - 1] + NZCount[ind], \textnormal{ for: } 0 < ind < m.
\end{equation}
$NZCount[ind]$ is the count of non-zero elements in column $ind$, with $ind$ between $0$ and $m$. Thus:
\nomenclature[A]{NNZ}{Number of non-zero elements of a given matrix}
\nomenclature[R]{M}{A generic matrix}
\nomenclature[U]{p}{For a given $M$ sparse matrix, $M_p$ is an array with the list of the cumulative count of non-zero entries in each column of $M$, utilized in the CSC storage method}
\nomenclature[U]{i}{For a given $M$ sparse matrix, the array $M_i$ is the list of coordinates of the non-zero entries in terms of row indexes for each column of $M$, utilized in the CSC storage method}
\nomenclature[U]{x}{For a given $M$ sparse matrix, the array $M_x$ is the list of values of the non-zero elements of the matrix $M$, utilized in the CSC storage method}
	\begin{equation}
	A_p = 
	\begin{bmatrix}
	0	&2	&5	&9	&10	&12
	\end{bmatrix}
	.
	\end{equation}
By definition, ${A_p}[0] = 0$. ${A_p}[1] = 2$, since there are two non-zero entries in the first column. ${A_p}[2] = 5$ since there are three non-zero entries in the second column and two in the first column. An so on. Finally, ${A_p}[5] = 12$ because there are 12 non-zero values in the columns 1 to 5, and thus in the whole matrix $A$. The NNZ of $A$ is the number of non-zero entries of the matrix $A$, and it always equals to $A_p[m]$, the last element of $A_p$. 

Next, the array $A_i$ is the list of coordinates of the non-zero entries in the matrix, in terms of row indexes, from the left to the right column. It considers the upmost row index equals to zero, as it is commonly done in computer sciences. Therefore:
	\begin{equation}
	A_i = 
	\begin{bmatrix}
	0	&1	&0	&2	&4	&1	&2	&3	&4	&2	&1	&4
	\end{bmatrix}
	.
	\end{equation}
${A_i}[0] = 0$ since the first element of the first column $A_[0][0] = 2$. ${A_i}[1] = 1$ since the second element of the first column $A_[0][1] = 3$. Since there are no other elements in the first column, ${A_i}[2] = 0$, since $A[1][0] = 3$. And so on. The last element of $A_i$, ${A_i}[1][1] = 4$ since 4 is the row index of the last non-zero number, $A_[4][4] = 1$.
	
Finally, the array $A_x$ is the list of values of the non-zero elements of the matrix $A$, starting from the top-left to the bottom-right corner of the matrix. Thus: 
	\begin{equation}
	A_x = 
	\begin{bmatrix}
	2	&3	&3	&-1	&4	&4	&-3	&1	&2	&2	&6	&1
	\end{bmatrix}
	.
	\end{equation}
Those three arrays together, $A_p$, $A_i$ and $A_x$ can store the data of the $A$ matrix. $A_p$ and $A_i$ are arrays of non-negative numbers since they only store counts and indexes. That simple fact alone can decrease the storage size of the $A$ matrix since integer variables are smaller than floating-point ones. $A_x$ is an array of the same type of variable that the matrix $A$ stores since it represents the non-zero elements of $A$. If $A$ is a matrix of floating-point variables, then $A_x$ is an array of floating-point variables. If $A$ is a matrix of integers, then $A_x$ stores integers. The dimension of $A_p$ is $m+1$, with $m$ equals to the number of columns of the matrix $A$. The dimensions of $A_i$ and $A_x$ are equal to the NNZ of the matrix $A$. 

This storing method can substantially reduce the amount of memory utilized in the operations with the matrix $A$ when compared to a simple, two-dimensional array representation of all the matrix elements. Some sparse linear systems are unpractical to be solved when storing with an uncompressed matrix. This difference in terms of efficiency is higher for matrices with higher sparsities. The next section will show how this storing method has been applied to compress the sparse matrix of this specific project.

\section{Representing the Coefficient Matrix of the Linear System of Equations with the CSC Method}

\autoref{appendices_vis} shows that the linear system of equations spawn by the Eq. \ref{matrix2} can be written as a matrix multiplication in the form:
\begin{align}
\label{appendix_smr_1}
A x = B,
\end{align}
illustrated by Figures \ref{fig:04} and \ref{fig:27}. The matrix $A$ is an heptadiagonal sparse matrix in that each of its diagonals is defined by the pressure coefficients in the Eq. \ref{matrix2}: $A^{\stackrel{(\nu)}{n+1}}_{i,j,k}$, $B^{\stackrel{(\nu)}{n+1}}_{i,j,k}$, $N^{\stackrel{(\nu)}{n+1}}_{i,j,k}$, $S^{\stackrel{(\nu)}{n+1}}_{i,j,k}$, $E^{\stackrel{(\nu)}{n+1}}_{i,j,k}$, $W^{\stackrel{(\nu)}{n+1}}_{i,j,k}$ and $C^{\stackrel{(\nu)}{n+1}}_{i,j,k}$.
This matrix can be stored in a compressed form by utilizing the method CSC. That consists in writing the matrix in terms of the three arrays $A_p$, $A_i$ and $A_x$, defined in the previous section. Advancing the simulation in time will not change the overall layout of the matrix $A$ in terms of the position of its non-zero elements. This is due to the nature of the problem: the number of pairwise interactions of each grid block in the reservoir does not change with time. Thus, the matrix $A$ will always have this heptadiagonal form in this problem, described by Figure \ref{fig:27}. That implies that the count and position of the non-zero elements of the matrix $A$ does not change with time, and neither do $A_p$ and $A_i$. Therefore, they only need to be calculated one time for each simulated model. On the other hand, the values of the non-zero elements can change with time. Hence, $A_x$ should be recalculated in each time step. 

As shown in the previous section, $A_p$ is an array of non-negative integers with the dimension equal to $m + 1$ where $m$ is the number of columns in the matrix $A$. As shown in the \ref{appendices_vis}, the matrix $A$ is square and its dimensions are $n \times n$ where $n$ is equaled to the number of blocks in the grid. Thus, the dimension of $A_p$ is equal to $n_x n_y n_z + 1$. By definition, ${A_p}[0] = 0$. Next, ${A_p}_{ind}$ is equal to the number of non-zero coefficients of pressure in Eq. \ref{matrix2} for the grid blocks of indexes $ind$ or below. This section will utilize as an example the model described by the Figure \ref{fig:26}, from \cite{Ertekin2001}, but a similar approach has been done for every model run in this project. For that example, the grid block of index $1$ has only $4$ non-zero coefficients, as shown by Eq. \ref{solution2}. Next, the grid block $2$ has $5$ non-zero coefficients, as can be seen in Eq. \ref{solution3}. Repeating this data collection process for all the grid blocks, it is possible to build the Table \ref{table_Ap}, from the number of pressure coefficients found in Eqs. \ref{solution2} to \ref{solution37}.
\begin{center}
	\begin{longtable}[htbp]{c c c}
		\caption{Count of the non-zero coefficients of the Eq. \ref{matrix2} for the grid blocks in the example described by Figure \ref{fig:26} from \cite{Ertekin2001}.}\\
		\toprule
		Index & Number of & Cumulative sum of the\\
		$ind$ &	non-zero coefficients	&	count of the non-zero coefficients\\
		\midrule
		1	&	4	&	4	\\
		2	&	5	&	9	\\
		3	&	5	&	14	\\
		4	&	4	&	18	\\
		5	&	5	&	23	\\
		6	&	6	&	29	\\
		7	&	6	&	35	\\
		8	&	5	&	40	\\
		9	&	4	&	44	\\
		10	&	5	&	49	\\
		11	&	5	&	54	\\
		12	&	4	&	58	\\
		13	&	5	&	63	\\
		14	&	6	&	69	\\
		15	&	6	&	75	\\
		16	&	5	&	80	\\
		17	&	6	&	86	\\
		18	&	7	&	93	\\
		19	&	7	&	100	\\
		20	&	6	&	106	\\
		21	&	5	&	111	\\
		22	&	6	&	117	\\
		23	&	6	&	123	\\
		24	&	5	&	128	\\
		25	&	4	&	132	\\
		26	&	5	&	137	\\
		27	&	5	&	142	\\
		28	&	4	&	146	\\
		29	&	5	&	151	\\
		30	&	6	&	157	\\
		31	&	6	&	163	\\
		32	&	5	&	168	\\
		33	&	4	&	172	\\
		34	&	5	&	177	\\
		35	&	5	&	182	\\
		36	&	4	&	186	\\
		\bottomrule
		\label{table_Ap}
	\end{longtable}
\end{center}
\vspace{-5.5ex}
Note that all the grid blocks shown in the table above have either 4, 5, 6, or 7 non-zero coefficients. This same fact can be seen in the rows of \ref{fig:27}, they also have from 4 to 7 non-zero elements. That is explained by the boundaries conditions of the problem: the transmissibilities are set to be zero at the reservoir boundaries. A grid block in the center of the reservoir normally has 7 coefficients in that equation: $A^{\stackrel{(\nu)}{n+1}}_{i,j,k}$, $B^{\stackrel{(\nu)}{n+1}}_{i,j,k}$, $N^{\stackrel{(\nu)}{n+1}}_{i,j,k}$, $S^{\stackrel{(\nu)}{n+1}}_{i,j,k}$, $E^{\stackrel{(\nu)}{n+1}}_{i,j,k}$, $W^{\stackrel{(\nu)}{n+1}}_{i,j,k}$ and $C^{\stackrel{(\nu)}{n+1}}_{i,j,k}$.
For the first grid block, the coefficients $B^{\stackrel{(\nu)}{n+1}}_{i,j,k}$, $S^{\stackrel{(\nu)}{n+1}}_{i,j,k}$ and $W^{\stackrel{(\nu)}{n+1}}_{i,j,k}$ are zero since those faces are in the boundaries of the reservoir. The number of coefficients of a given grid block equals to $7$ minus the number of faces that are on the limits of the reservoir. For the specific geometry of this project, a grid block can only have from zero to three faces in a boundary. Therefore, it is only possible to have from 4 to 7 non-zero coefficients. The third column in Table \ref{table_Ap} is the cumulative sum of the count of those coefficients for each grid block. That last column is equals to $A_p$, which written in computer sciences notation is:
\begin{equation}
\label{Ap}
A_p = 
\begin{bmatrix}
0	&4	&9	&14	&18	&23	&29	&35 &40 &44 &49 &54 &... &182 &186
\end{bmatrix}
.
\end{equation}
$A_i$ is the list of the row indexes of the non-zero coefficients of the matrix $A$. For this problem, it represents the grid block indexes of the pressures utilized in the \ref{matrix2} equation for a given grid block, in ascending order. For the grid block $1$ of this example, those are $1$, $2$, $5$, and $13$, as can be seen in Eq. \ref{solution2}. For the grid block $2$, those are: 1,2,3,6 and 14, according to Eq. \ref{solution3}. Table \ref{table_AiAx} shows those pressure indexes for each one of the 36 grid blocks of this example. The table also shows the coefficients of those pressures. Those are equal to $A_i$ and $A_x$, which in computer sciences notation are:
\begin{equation}
\label{Ai}
A_i = 
\begin{bmatrix}
0	&1	&4	&12	&0	&1	&2	&5 &13 &0 &... &23 &31 &34 &35
\end{bmatrix}
,
\end{equation}
and:
\begin{equation}
\label{Ax}
A_x = 
\begin{bmatrix}
C^{\stackrel{(\nu)}{n+1}}_{1} &E^{\stackrel{(\nu)}{n+1}}_{1} &N^{\stackrel{(\nu)}{n+1}}_{1} &B^{\stackrel{(\nu)}{n+1}}_{1} &W^{\stackrel{(\nu)}{n+1}}_{2} &... &A^{\stackrel{(\nu)}{n+1}}_{36} &S^{\stackrel{(\nu)}{n+1}}_{36} &W^{\stackrel{(\nu)}{n+1}}_{36} &C^{\stackrel{(\nu)}{n+1}}_{36}
\end{bmatrix}
.
\end{equation}
Hence, the matrix resulting from Eqs. \ref{solution2} to \ref{solution37} can be stored with the CSC method by utilizing the arrays $A_p$, $A_i$ and $A_x$ defined by Eqs. \ref{Ap}, \ref{Ai} and \ref{Ax}.
\begin{center}
	\begin{longtable}[htbp]{c c c}
		\caption{Count of the non-zero coefficients of the Eq. \ref{matrix2} for the grid blocks in the example described by Figure \ref{fig:26} from \cite{Ertekin2001}.}\label{table_AiAx} \\
		\toprule
		Index & Pressure indexes & Pressure coefficients\\
		$ind$ &	in Eq. \ref{matrix2}	&	in Eq. \ref{matrix2}\\
		\midrule
		\endfirsthead
		
		\multicolumn{3}{c}%
		{{\tablename\ \thetable{} -- continued from previous page}} \\
		\toprule
		Index & Pressure indexes & Pressure coefficients\\
		$ind$ &	in Eq. \ref{matrix2}	&	in Eq. \ref{matrix2}\\
		\midrule
		\endhead
		
		\hline \multicolumn{3}{r}{{Continued on next page}} \\ \hline
		\endfoot
		
		\bottomrule
		\endlastfoot
	
		1	&	1,2,5,13	&	$C^{\stackrel{(\nu)}{n+1}}_{1}$, $E^{\stackrel{(\nu)}{n+1}}_{1}$, $N^{\stackrel{(\nu)}{n+1}}_{1}$, $B^{\stackrel{(\nu)}{n+1}}_{1}$ 	\\
		2	&	1,2,3,6,14	&	$W^{\stackrel{(\nu)}{n+1}}_{2}$, $C^{\stackrel{(\nu)}{n+1}}_{2}$, $E^{\stackrel{(\nu)}{n+1}}_{2}$, $N^{\stackrel{(\nu)}{n+1}}_{2}$, $B^{\stackrel{(\nu)}{n+1}}_{2}$	\\
		3	&	2,3,4,7,15	&	$W^{\stackrel{(\nu)}{n+1}}_{3}$, $C^{\stackrel{(\nu)}{n+1}}_{3}$, $E^{\stackrel{(\nu)}{n+1}}_{3}$, $N^{\stackrel{(\nu)}{n+1}}_{3}$, $B^{\stackrel{(\nu)}{n+1}}_{3}$	\\
		4	&	3,4,8,16	&	$W^{\stackrel{(\nu)}{n+1}}_{4}$, $C^{\stackrel{(\nu)}{n+1}}_{4}$, $E^{\stackrel{(\nu)}{n+1}}_{4}$, $B^{\stackrel{(\nu)}{n+1}}_{4}$	\\
		5	&	1,5,6,9,17	&	$S^{\stackrel{(\nu)}{n+1}}_{5}$, $C^{\stackrel{(\nu)}{n+1}}_{5}$, $E^{\stackrel{(\nu)}{n+1}}_{5}$, $N^{\stackrel{(\nu)}{n+1}}_{5}$, $B^{\stackrel{(\nu)}{n+1}}_{5}$	\\
		6	&	2,5,6,7,10,18	&	$S^{\stackrel{(\nu)}{n+1}}_{6}$,$W^{\stackrel{(\nu)}{n+1}}_{6}$,$C^{\stackrel{(\nu)}{n+1}}_{6}$, $E^{\stackrel{(\nu)}{n+1}}_{6}$, $N^{\stackrel{(\nu)}{n+1}}_{6}$, $B^{\stackrel{(\nu)}{n+1}}_{6}$	\\
		7	&	3,6,7,8,11,19	&	$S^{\stackrel{(\nu)}{n+1}}_{7}$,$W^{\stackrel{(\nu)}{n+1}}_{7}$,$C^{\stackrel{(\nu)}{n+1}}_{7}$, $E^{\stackrel{(\nu)}{n+1}}_{7}$, $N^{\stackrel{(\nu)}{n+1}}_{7}$, $B^{\stackrel{(\nu)}{n+1}}_{7}$	\\
		8	&	4,7,8,12,20	&	$S^{\stackrel{(\nu)}{n+1}}_{8}$,$W^{\stackrel{(\nu)}{n+1}}_{8}$, $C^{\stackrel{(\nu)}{n+1}}_{8}$, $N^{\stackrel{(\nu)}{n+1}}_{8}$, $B^{\stackrel{(\nu)}{n+1}}_{8}$	\\
		9	&	5,9,10,21	&	$S^{\stackrel{(\nu)}{n+1}}_{9}$, $C^{\stackrel{(\nu)}{n+1}}_{9}$, $E^{\stackrel{(\nu)}{n+1}}_{9}$, $B^{\stackrel{(\nu)}{n+1}}_{9}$	\\
		10	&	6,9,10,11,22	&	$S^{\stackrel{(\nu)}{n+1}}_{10}$,$W^{\stackrel{(\nu)}{n+1}}_{10}$, $C^{\stackrel{(\nu)}{n+1}}_{10}$, $E^{\stackrel{(\nu)}{n+1}}_{10}$, $B^{\stackrel{(\nu)}{n+1}}_{10}$	\\
		11	&	7,10,11,12,23	&	$S^{\stackrel{(\nu)}{n+1}}_{11}$,$W^{\stackrel{(\nu)}{n+1}}_{11}$, $C^{\stackrel{(\nu)}{n+1}}_{11}$, $E^{\stackrel{(\nu)}{n+1}}_{11}$, $B^{\stackrel{(\nu)}{n+1}}_{11}$	\\
		12	&	8,11,12,24	&	$S^{\stackrel{(\nu)}{n+1}}_{12}$,$W^{\stackrel{(\nu)}{n+1}}_{12}$, $C^{\stackrel{(\nu)}{n+1}}_{12}$,  $B^{\stackrel{(\nu)}{n+1}}_{12}$	\\
		13	&	1,13,14,17,25	&	$A^{\stackrel{(\nu)}{n+1}}_{13}$,$C^{\stackrel{(\nu)}{n+1}}_{13}$, $E^{\stackrel{(\nu)}{n+1}}_{13}$, $N^{\stackrel{(\nu)}{n+1}}_{13}$, $B^{\stackrel{(\nu)}{n+1}}_{13}$	\\
		14	&	2,13,14,15,18,26	&	$A^{\stackrel{(\nu)}{n+1}}_{14}$,$W^{\stackrel{(\nu)}{n+1}}_{14}$, $C^{\stackrel{(\nu)}{n+1}}_{14}$, $E^{\stackrel{(\nu)}{n+1}}_{14}$, $N^{\stackrel{(\nu)}{n+1}}_{14}$, $B^{\stackrel{(\nu)}{n+1}}_{14}$	\\
		15	&	3,14,15,16,19,27	&	$A^{\stackrel{(\nu)}{n+1}}_{15}$,$W^{\stackrel{(\nu)}{n+1}}_{15}$, $C^{\stackrel{(\nu)}{n+1}}_{15}$, $E^{\stackrel{(\nu)}{n+1}}_{15}$, $N^{\stackrel{(\nu)}{n+1}}_{15}$, $B^{\stackrel{(\nu)}{n+1}}_{15}$	\\
		16	&	4,15,16,20,28	&	$A^{\stackrel{(\nu)}{n+1}}_{16}$,$W^{\stackrel{(\nu)}{n+1}}_{16}$, $C^{\stackrel{(\nu)}{n+1}}_{16}$, $E^{\stackrel{(\nu)}{n+1}}_{16}$, $B^{\stackrel{(\nu)}{n+1}}_{16}$	\\
		17	&	5,13,17,18,21,29	&	$A^{\stackrel{(\nu)}{n+1}}_{17}$,$S^{\stackrel{(\nu)}{n+1}}_{17}$, $C^{\stackrel{(\nu)}{n+1}}_{17}$, $E^{\stackrel{(\nu)}{n+1}}_{17}$, $N^{\stackrel{(\nu)}{n+1}}_{17}$, $B^{\stackrel{(\nu)}{n+1}}_{17}$	\\
		18	&	6,14,17,18,19,22,30	&	$A^{\stackrel{(\nu)}{n+1}}_{18}$,$S^{\stackrel{(\nu)}{n+1}}_{18}$,$W^{\stackrel{(\nu)}{n+1}}_{18}$, $C^{\stackrel{(\nu)}{n+1}}_{18}$, $E^{\stackrel{(\nu)}{n+1}}_{18}$, $N^{\stackrel{(\nu)}{n+1}}_{18}$, $B^{\stackrel{(\nu)}{n+1}}_{18}$	\\
		19	&	7,15,18,19,20,23,31	&	$A^{\stackrel{(\nu)}{n+1}}_{19}$,$S^{\stackrel{(\nu)}{n+1}}_{19}$,$W^{\stackrel{(\nu)}{n+1}}_{19}$, $C^{\stackrel{(\nu)}{n+1}}_{19}$, $E^{\stackrel{(\nu)}{n+1}}_{19}$, $N^{\stackrel{(\nu)}{n+1}}_{19}$, $B^{\stackrel{(\nu)}{n+1}}_{19}$	\\
		20	&	8,16,19,29,24,32	&	$A^{\stackrel{(\nu)}{n+1}}_{20}$,$S^{\stackrel{(\nu)}{n+1}}_{20}$,$W^{\stackrel{(\nu)}{n+1}}_{20}$, $C^{\stackrel{(\nu)}{n+1}}_{20}$, $N^{\stackrel{(\nu)}{n+1}}_{20}$, $B^{\stackrel{(\nu)}{n+1}}_{20}$	\\
		21	&	9,17,12,22,33	&	$A^{\stackrel{(\nu)}{n+1}}_{21}$,$S^{\stackrel{(\nu)}{n+1}}_{21}$, $C^{\stackrel{(\nu)}{n+1}}_{21}$, $E^{\stackrel{(\nu)}{n+1}}_{21}$, $B^{\stackrel{(\nu)}{n+1}}_{21}$	\\
		22	&	10,18,21,22,23,34	&	$A^{\stackrel{(\nu)}{n+1}}_{22}$,$S^{\stackrel{(\nu)}{n+1}}_{22}$,$W^{\stackrel{(\nu)}{n+1}}_{22}$, $C^{\stackrel{(\nu)}{n+1}}_{22}$, $E^{\stackrel{(\nu)}{n+1}}_{22}$, $B^{\stackrel{(\nu)}{n+1}}_{22}$	\\
		23	&	11,19,22,23,24,35	&	$A^{\stackrel{(\nu)}{n+1}}_{23}$,$S^{\stackrel{(\nu)}{n+1}}_{23}$,$W^{\stackrel{(\nu)}{n+1}}_{23}$, $C^{\stackrel{(\nu)}{n+1}}_{23}$, $E^{\stackrel{(\nu)}{n+1}}_{23}$, $B^{\stackrel{(\nu)}{n+1}}_{23}$	\\
		24	&	12,20,23,24,36	&	$A^{\stackrel{(\nu)}{n+1}}_{24}$,$S^{\stackrel{(\nu)}{n+1}}_{24}$,$W^{\stackrel{(\nu)}{n+1}}_{24}$, $C^{\stackrel{(\nu)}{n+1}}_{24}$,  $B^{\stackrel{(\nu)}{n+1}}_{24}$	\\
		25	&	13,25,26,29	&	$A^{\stackrel{(\nu)}{n+1}}_{25}$,$C^{\stackrel{(\nu)}{n+1}}_{25}$, $E^{\stackrel{(\nu)}{n+1}}_{25}$, $N^{\stackrel{(\nu)}{n+1}}_{25}$	\\
		26	&	14,26,27,30	&	$A^{\stackrel{(\nu)}{n+1}}_{26}$,$W^{\stackrel{(\nu)}{n+1}}_{26}$, $C^{\stackrel{(\nu)}{n+1}}_{26}$, $E^{\stackrel{(\nu)}{n+1}}_{26}$, $N^{\stackrel{(\nu)}{n+1}}_{26}$	\\
		27	&	15,26,27,28,31	&	$A^{\stackrel{(\nu)}{n+1}}_{27}$,$W^{\stackrel{(\nu)}{n+1}}_{27}$, $C^{\stackrel{(\nu)}{n+1}}_{27}$, $E^{\stackrel{(\nu)}{n+1}}_{27}$, $N^{\stackrel{(\nu)}{n+1}}_{27}$	\\
		28	&	16,27,28,32	&	$A^{\stackrel{(\nu)}{n+1}}_{28}$,$W^{\stackrel{(\nu)}{n+1}}_{28}$, $C^{\stackrel{(\nu)}{n+1}}_{28}$, $E^{\stackrel{(\nu)}{n+1}}_{28}$	\\
		29	&	17,25,29,30,33	&	$A^{\stackrel{(\nu)}{n+1}}_{29}$,$S^{\stackrel{(\nu)}{n+1}}_{29}$, $C^{\stackrel{(\nu)}{n+1}}_{29}$, $E^{\stackrel{(\nu)}{n+1}}_{29}$, $N^{\stackrel{(\nu)}{n+1}}_{29}$	\\
		30	&	18,26,29,30,31,34	&	$A^{\stackrel{(\nu)}{n+1}}_{30}$,$S^{\stackrel{(\nu)}{n+1}}_{30}$,$W^{\stackrel{(\nu)}{n+1}}_{30}$, $C^{\stackrel{(\nu)}{n+1}}_{30}$, $E^{\stackrel{(\nu)}{n+1}}_{30}$, $N^{\stackrel{(\nu)}{n+1}}_{30}$	\\
		31	&	19,27,30,31,32,35	&	$A^{\stackrel{(\nu)}{n+1}}_{31}$,$S^{\stackrel{(\nu)}{n+1}}_{31}$,$W^{\stackrel{(\nu)}{n+1}}_{31}$, $C^{\stackrel{(\nu)}{n+1}}_{31}$, $E^{\stackrel{(\nu)}{n+1}}_{31}$, $N^{\stackrel{(\nu)}{n+1}}_{31}$	\\
		32	&	20,28,31,32,36	&	$A^{\stackrel{(\nu)}{n+1}}_{32}$,$S^{\stackrel{(\nu)}{n+1}}_{32}$,$W^{\stackrel{(\nu)}{n+1}}_{32}$, $C^{\stackrel{(\nu)}{n+1}}_{32}$, $N^{\stackrel{(\nu)}{n+1}}_{32}$	\\
		33	&	21,29,33,34	&	$A^{\stackrel{(\nu)}{n+1}}_{33}$,$S^{\stackrel{(\nu)}{n+1}}_{33}$, $C^{\stackrel{(\nu)}{n+1}}_{33}$, $E^{\stackrel{(\nu)}{n+1}}_{33}$, $B^{\stackrel{(\nu)}{n+1}}_{33}$	\\
		34	&	22,30,33,34,35	&	$A^{\stackrel{(\nu)}{n+1}}_{34}$,$S^{\stackrel{(\nu)}{n+1}}_{34}$,$W^{\stackrel{(\nu)}{n+1}}_{34}$, $C^{\stackrel{(\nu)}{n+1}}_{34}$, $E^{\stackrel{(\nu)}{n+1}}_{34}$	\\
		35	&	23,31,34,35,36	&	$A^{\stackrel{(\nu)}{n+1}}_{35}$,$S^{\stackrel{(\nu)}{n+1}}_{35}$,$W^{\stackrel{(\nu)}{n+1}}_{35}$, $C^{\stackrel{(\nu)}{n+1}}_{35}$, $E^{\stackrel{(\nu)}{n+1}}_{35}$	\\
		36	&	24,32,35,36	&	$A^{\stackrel{(\nu)}{n+1}}_{36}$,$S^{\stackrel{(\nu)}{n+1}}_{36}$,$W^{\stackrel{(\nu)}{n+1}}_{36}$, $C^{\stackrel{(\nu)}{n+1}}_{36}$ \\
\end{longtable}
\end{center}


