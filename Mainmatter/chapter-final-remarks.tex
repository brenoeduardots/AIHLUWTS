\chapter{Final Remarks}

The main goal of this project has been to analyze the results of well tests in reservoir models both before and after their upscaling.
%
This analysis requires the utilization of a reservoir simulator, which was also developed in this project. 
%
The first part of this document shows the development of this reservoir simulator. 
%
Then, the basic theory behind upscaling and well testing is presented in its second part. 
%
The third and final part of the document shows the construction of the fine-grid and upscaled models and the results and analysis of their well test simulations.

The reservoir simulator has been developed for enabling the simulation of single-phase flow of compressible fluid in a tridimensional, block-centered grid representing a sealed reservoir with a well in its center. 
%
Chapter \ref{chapter-mathematical-formulation} shows the mathematical formulation of this simulator. 
%
The discretization has been approached by the finite-difference method known as BTCS (Backwards Time, Central Space) and the linearization has been done by the method described by \cite{Ertekin2001} as the simple iteration of the transmissibility terms. 
%
This numerical model is described in more detail in Chapter \ref{chapter-numerical-formulation}. 
%
The well model has been developed for taking into account the effects of multiple perforated layers, as described in Chapter \ref{chapter-well-representation}, a necessary approach when analyzing upscaling. 
%
The simulator has been validated by comparing its results to simulations done by industry-standard software. 
%
In conclusion, the software obtained a high accuracy in its results, as shown in Chapter \ref{chapter-solution}, and the validation was a success.

Next, Chapter \ref{chapter-upscaling} shows the fundamentals of upscaling, highlighting the simple volumetric averaging and the arithmetic-harmonic averaging upscaling, the techniques utilized in this project for the scaling up of porosity and permeability, respectively. 
%
Chapter \ref{chapter-well-testing} shows the basics of well testing and the plots analyzed in this project, the bottom-hole pressure, pressure drop, and Bourdet derivative curves.

At last, the analysis of the impacts of heterogeneity losses due to upscaling in well test simulations are shown in Chapter \ref{chapter-AIHLUWTS}. 
%
Firstly, four fine-grid reservoir models have been created, cases 1, 3, 5, and 7. 
%
The only difference between one of those models to another is the dispersion of porosity and permeability. 
%
The first case is homogeneous and each other case has a different value of standard deviations. 
%
Then, four other models, cases 2, 4, 6, and 8, have been generated by the upscaling of those fine-grid models. 
%
Appendix \ref{appendix-models-descriptions} shows the petrophysical distributions of all those models in more detail. 
%
All those eight models have been put under flow simulation for a drawdown stage of a well test, and the results of bottom-hole pressure, pressure drop, and Bourdet derivative have been output. 
%
An analysis of the BHP plots shows that the differences in the curves of bottom-hole pressure for the fine-grid and upscaled models are higher for the cases with more porosity-permeability dispersion. 
%
This is also true for the pressure drop and Bourdet derivative, the differences of those curves for the fine-grid and upscaled models are also higher for the cases more heterogeneous. 
%
The final part of the Bourdet derivative curves have slightly different slopes for the models before and after their upscaling. 
%
This might induce biases in a pressure transient analysis, considering that this slope is often utilized for inferring dynamic characteristics of the well, such as the flow regime. 
%
Nevertheless, a more profound analysis is required.

\section{Future Suggestions}

Future suggestions for this line of study can be divided into two branches: the analysis of the effects of upscaling in well test simulations and the improvement of the reservoir simulator developed in this project.
%
For the first, some suggestions for future projects include:
%
\begin{itemize}
	\item Verify if the difference in fine-grid and upscaled models could generate misinterpretations in pressure transient analysis by evaluating slopes in the Bourdet derivative curve. 
	
	\item Perform the evaluation above for other heterogeneity scenarios other than white noise, such as fractures, high-permeability layers, and channels.
	
	\item Analyze the effects of wellbore storage in the Bourdet derivative curves for fine-grid and upscaled models.
	
	\item Perform the evaluations above, utilizing different techniques of upscaling, such as harmonic averaging, arithmetic averaging, harmonic- arithmetic averaging, and flow-base methods.
\end{itemize}
\noindent
%
Some suggestions for improving the simulator developed in this project are:

\begin{itemize}
	\item Expand the simulator for handling other internal and external boundary conditions.
	
	\item Expand the simulator for allowing multiple wells and well control groups.
	
	\item Expand the mathematical/numerical model for multiphase flow, for a black oil or compositional model.
	
	\item Improve the well model for allowing horizontal wells.
	
	\item Allow the utilization of grids other than block centered, Cartesian grid. Some examples include a hybrid grid (Cartesian with cylindrical coordinates in the surroundings of the wells), Voronoi grid, and corner-point grid.
	
	\item Enable the utilization of the net-to-gross ratio for allowing the modeling of more advanced geological features.
\end{itemize}


