\chapter{Final Remarks}
A reservoir simulator has been developed, obtaining good accuracy when comparing its results to industry standard simulators. Four synthetic sandstone models have been created considering different levels of heterogeneities: homogeneous and heterogeneous with different standard deviations. Those models have been upscaled in four new models, by utilizing arithmetic-harmonic upscaling. The results of bottom hole pressure, pressure drawdown, and Bourdet derivative have been compared for the eight models. Results have shown that the higher standard deviations imply less representative upscaling. Some suggestions for future projects include: 

$\bullet$ Compare the pressure drop and Bourdet derivative results with the analytical solution.

$\bullet$ Analyze the effects of wellbore storage effects in the Bourdet derivative results for fine and upscaled models.

$\bullet$ Evaluate different approaches for upscaling, such as harmonic averaging, arithmetic averaging, and flow-base methods.

$\bullet$ Analyze the impact of upscaling in models with different heterogeneities, such as super-k layers, fractured reservoirs, and channels.

$\bullet$ Expand the simulator for handling other internal and externals boundary conditions.

$\bullet$ Expand the simulator for allowing horizontal and multilateral wells.

$\bullet$ Expand the simulator for allowing multiple wells.

$\bullet$ Enable to utilize a cylindrical coordinate system in the simulator. 

$\bullet$ Enable to utilize a hybrid grid in the near-wellbore region in the simulator. 

